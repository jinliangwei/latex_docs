\documentclass{beamer}
\usepackage{beamerthemesplit} % new 

\usetheme{default}
\useoutertheme{infolines}
%\setbeamertemplate{navigation symbols}{gd} 

\begin{document}


\title{Concurrent LRU Based on Concurrent Hash Table} 
\author{Jinliang Wei} 
\date{\today} 

\frame{\titlepage} 

\frame{\frametitle{Table of contents}\tableofcontents} 

\section{Process Storage}

\begin{frame}
\frametitle{Process Storage Basics}
ProcessStorage caches a set of rows. The maximum number of row can be stored is
\emph{capacity}. Caching means that 
\begin{enumerate}
\item The storage does not necessarily contains the row that its clients ask for;
\item Resident rows may get evicted anytime unless it's properly protected by 
reference counting. 
\end{enumerate}

\end{frame}

\begin{frame}
\frametitle{Process Storage API - \texttt{Find()}}
Set the accessor to point to the row specified by 
  the row index. Once an accessor is set, it is guaranteed to remain valid until
 the accessor is destroyed. ``Remain valid'' means that reading
from the accessor always returns some value. However, whether or not the value 
satisfies the consistency requirements depends on the specific consistency model.
\begin{itemize}
\item SSP: Freshness of the value read from accessor is not guaranteed across iterations.
  Generally, accessors from a previous iteration should not be reused.
\item VAP: accessors remains valid throughout the execution.
\end{itemize}
\end{frame}

\begin{frame}
\frametitle{Process Storage API - \texttt{Insert()}}
Insert a client row (including metadata). Optionally it returns the evicted row and
provides an accessor to the inserted row. \texttt{Insert()} assumes the number of rows
being referenced never reaches \emph{capacity} of the ProcessStorage.
The insertion logic is as follows:
\begin{enumerate}
\item If the row index does not exist in the ProcessStorage, the row is inserted
 into an empty slot. If ProcessStorage cannot find an empty slot, it may evict an
 resident row that have zero reference count.
\item If the row index already exists in the table. It will be replaced by the 
  inserted row regardless the reference count. The replacement guarantees that 
  if the resident row is being referenced, it will remain valid until all references
  are cleared.
\end{enumerate}

\end{frame}

\begin{frame}
\frametitle{Process Storage Thread Safety}
\begin{itemize}
\item \texttt{Find()} and \texttt{Insert()} may be executed concurrently.
\item \texttt{Find()} and \texttt{Insert()} on the same row index are mutually exclusive.
\item Mutual exclusiveness implies that if a \texttt{Find()} returns ``Not Found''
for a row index, the row does not exist in ProcessStorage before \texttt{Find()} is 
invoked.
\item Write (\texttt{Insert()}) appears to be atomic. That means if there is a 
``happen-before''relation between a write and a read (\texttt{Find()}), the 
result of the write is guaranteed to be visiable.
\end{itemize}
\end{frame}

\begin{frame}
\frametitle{Process Storage Assumptions and Guarantees}
\begin{itemize}
\item Process Storage is initialized with a \emph{capacity} number. It 
guarantees that the number of rows cached in ProcessStorage does not exceed 
\emph{capacity} (with a minor exception as explained later).

\item Rows are accessed via accessors which essentially hold a reference to
a row. If too many (equal to \emph{capacity}) rows are being accessed 
(references are held to them), the process storage won't be able to evict
any row so it cannot insert a new row.
\end{itemize}
\end{frame}

\section{Per-Row Reference Count}

\begin{frame}
Per-Row Reference Count
\end{frame}

\begin{frame}
\frametitle{Major Data Structures}
\begin{itemize}
\item Concurrent hash table: support concurrent insert, look-up and delete; 
support dynamic resizing; does not support concurrent traversal. The hash table
maps row index to client row pointer.
\item Striped lock: each lock guards a disjoint subset of row indices. A lock 
guarantees exclusive accesses to the object stored in the hash table for operations
that may alter the validity of the object and operations that requires the validity 
to remain unchanged throughout the course of excution. The object that's 
stored in the hash table is pointer, the striped lock guarantees exclusive 
access to the memory that's pointed to by the pointer via ProcessStorage. However,
the client row also accessed outside the ProcessStorage, ProcessStorage assumes
concurent operations from outside on the client row is properly handled by the client 
row.
\item An integer to denote the number of occupied slots in the hash table.
\end{itemize}
\end{frame}

\begin{frame}
\frametitle{Consistency Requirements for Concurrent Hash Table}
\begin{itemize}
\item Reads and Writes on distinct keys may happen concurrently.
\item The order of operations of a single thread is preserved.
\item If there's a ``happen-before'' relation between a write and a read on the 
  same key, the result of the write is guaranteed to be seen by the read. The 
``happend-before'' may be established via operations outside the hash table. So 
that is different from sequential consisitency of the concurrent hash table.
\end{itemize}
\end{frame}

\begin{frame}
\frametitle{Concurrent Cuckoo Hash Table}
\begin{itemize}
\item Although not mentioned in its docuemnt, libcuckoo provides sequential 
consistency with one additional constrain: the ordering respects the ordering
operations outside the hash table. This satisfiled the requirement.
\end{itemize}
\end{frame}

\begin{frame}
\frametitle{CLOCK LRU Interface}
The CLOCK LRU logic shares access to the striped lock and the hash table, and 
provides the following APIs:

\begin{itemize}
\item \texttt{FindOneToEvict()}: return a row index to caller which is suggested
  to be evicted. The row may be actually evicted or not depending on other logic
  of ProcessStorage, e.g. reference count. The caller thread of 
  \texttt{FindOneToEvict()} must invoke \texttt{Evict()} or \texttt{NoEvict()} 
  before invoking any other operation CLOCK LRU. However, all operations of
  CLOCK LRU are thread-safe.

\item \texttt{Evict()}: inform the CLOCK LRU that a row has been evicted. The
number of occupied LRU slots is decremented by 1.

\item \texttt{NoEvict()}: notify the CLOCK LRU that the suggested row is not
evicted. There's no change to the number of occupied slots.

\end{itemize}
\end{frame}

\begin{frame}
  \frametitle{CLOCK LRU High Level Idea}
\begin{itemize}
\item CLOCK LRU suggests rows to ProcessStorage for eviction. The eviction makes 
assumption about ProcessStorage and provides certain guarantees.
\item The LRU logic keeps track of number $C$ rows, where $C$ is the cache capacity. We
refer to such space as LRU slots for rows. 
\end{itemize}
\end{frame}

\begin{frame}
\frametitle{CLOCK LRU Interface cont.}
\begin{itemize}
\item \texttt{Insert()}: notify the LRU logic that a row is resident in the 
  hash table. A row cannot become a cadidate for eviction until \texttt{Insert()}
  is invoked. Once succeed, the number of occupied LRU slots is incremented by 1.

\item \texttt{Reference()}: notify the LRU logic that a slot is recently accessed.
A parameter to this function is the slot number. LRU takes this a hint. It does 
not access striped lock.
\end{itemize}
\end{frame}

\begin{frame}
\frametitle{CLOCK LRU Assumptions}
\begin{itemize}
\item \texttt{Insert()} is invoked only if the number of occupied slots plus the 
  number of concurrent Insert() is strictly below cache capacity.
\item \texttt{Evict()} is invoked only if the numer of occupied slots minus the 
  number of concurrent Evict() is strictly larger than 0.
\end{itemize}
\end{frame}

\begin{frame}
\frametitle{ProcessStorage Implementation - \texttt{Find()}}
\begin{enumerate}
\item \underline{Acquire the striped lock based on row index}.
\item Find the row in the concurrent hash table.
\item Atomically increment the reference count of that row.
\item Set the accessor.
\item Inform the CLOCK LRU.
\item \underline{Release the lock}.
\end{enumerate}
\end{frame}

\begin{frame}
\frametitle{ProcessorStorage Implementation - \texttt{Insert()}}
\begin{enumerate}
\item Atomically fetch and add 1 to the number of occupied slots.
\item If the number of occupied slots exceeds \emph{capacity} of the
ProcessStorage,
\begin{enumerate}
\item Loop until break
\item Get a row from CLOCK LRU to evict.
\item \underline{Acquire the striped lock for the row to be evicted}.
\item Find that row (a pointer) from the hash table.
\item Check that row's reference count, if it is non-zero, inform CLOCK LRU, 
  \underline{release the lock}, repeat.
\item If it is zero, clear it from the hash table, destroy the row,  inform CLOCK LRU,
  \underline{releaset the lock} then break from the loop.
\item Atomically fetch and subtract 1 from the number of occupied slots.
\end{enumerate}
\item \underline{Acquire the striped lock for the row to be inserted}.
\item Find the row in concurrent hash table.
\item If the row exists, swap (including meta data and row data).
\item Otherwise, insert the row into the hash table.
\item \underline{Releaset the lock}.
\end{enumerate}
\end{frame}

\begin{frame}
\frametitle{Correctness Justification}
\begin{itemize}
\item The striped lock guarantees the accesses to \texttt{Insert()} and
\texttt{Find()} are mutually exclusive. The requirements of the concurrent hash
table and the ordering provided by striped lock guarantees the accesses are
sequential.
\item Lock on \texttt{Insert()} is necessary to ensure the previous insert on the
concurrent hash table is seen by a later operation on the same key. 
\end{itemize}
\end{frame}

\begin{frame}
\frametitle{Major Data Structures - CLOCK LRU}
CLOCK LRU logic, which contains the following data structures (bitsets and
 array are sized to capacity):
\begin{itemize}
\item A bitset of eviction bits: each bit denotes if a row can be evicted. If 
the bit is set, the slot is evictable otherwise it is not. This particular 
representation is to accomodate the operations of \texttt{std::atomic\_flag}.
\item A bitset of occupation bits: each bit denotes if the slot is occupied by a
resident row. If the bit is set, it denotes that the mapping from slot number
to row index points to a resident row in the hash table.
\item An array of integers mapping slot number to row index. This is needed when
finding a row to evict.
\item An thread-safe queue to store empty slots.
\begin{itemize}
\item ``Thread-safe'' indicates that operations on that queue is exclusive.
\end{itemize}
\end{itemize}

\end{frame}
\begin{frame}
\frametitle{Regarding Occupation Bits}
\begin{itemize}
\item The occupation bits are concerned about two operations: \texttt{Insert}, 
  \texttt{FindOneToEvict} and \texttt{Evict}.
\item Eviction cannot happen to an empty slot and insertion cannot happen to an 
occupied slot.
\end{itemize}
\end{frame}

\begin{frame}
\frametitle{CLOCK LRU Implementation - \texttt{FindOneToEvict()}}

\begin{enumerate}
\item Atomically fetch the value of the 
\end{enumerate}
\end{frame}

\section{Grouped Reference Count}

\begin{frame}
Grouped Reference Count
\end{frame}


\begin{frame}
\frametitle{Main Idea}

Store reference count outside of row object (possibly in LRU cache) to allow 
double-checked locking when access a resident row.

\end{frame}


\end{document}

